\documentclass[paper=a4, fontsize=11pt]{scrartcl} % A4 paper and 11pt font size

\usepackage[T1]{fontenc} % Use 8-bit encoding that has 256 glyphs
%\usepackage{fourier} % Use the Adobe Utopia font for the document - comment this line to return to the LaTeX default
\usepackage[italian]{babel} % English language/hyphenation
\usepackage[utf8]{inputenc}
\usepackage{amsmath,amsfonts,amsthm} % Math packages
\usepackage{graphicx}
\usepackage{wrapfig}
\usepackage{caption}
\usepackage{braket}
\usepackage{subcaption}
\usepackage[labelsep=endash]{caption}
\renewcommand{\figurename}{Figura}
\renewcommand{\tablename}{Tabella}
\usepackage{amssymb}
\usepackage[makeroom]{cancel}
\usepackage{mathtools}
\usepackage{float}
\usepackage{lipsum}
%\usepackage{pgfplots}
%\pgfplotsset{/pgf/number format/use comma,compat=newest}
\renewcommand\floatpagefraction{2.5}
\renewcommand\topfraction{2.5}
\usepackage{booktabs}
% \usepackage{array} or \usepackage{dcolumn}
\usepackage[table]{xcolor}
\usepackage[export]{adjustbox}
\usepackage{tikz}
\usepackage{multicol}
\usepackage {geometry}
%\pgfplotsset{width=7cm,compat=newest}
\setlength{\arrayrulewidth}{0.2mm}
\setlength{\tabcolsep}{35pt}
\renewcommand{\arraystretch}{0.35}
%\usepackage{lipsum} % Used for inserting dummy 'Lorem ipsum' text into the template

\usepackage{sectsty} % Allows customizing section commands
\allsectionsfont{\centering \normalfont\scshape} % Make all sections centered, the default font and small caps

\usepackage{fancyhdr} % Custom headers and footers
\pagestyle{fancyplain} % Makes all pages in the document conform to the custom headers and footers
\fancyhead{} % No page header - if you want one, create it in the same way as the footers below
\fancyfoot[L]{} % Empty left footer
\fancyfoot[C]{} % Empty center footer
\fancyfoot[R]{\thepage} % Page numbering for right footer
\renewcommand{\headrulewidth}{0pt} % Remove header underlines
\renewcommand{\footrulewidth}{0pt} % Remove footer underlines
\setlength{\headheight}{13.6pt} % Customize the height of the header

\numberwithin{equation}{section} % Number equations within sections (i.e. 1.1, 1.2, 2.1, 2.2 instead of 1, 2, 3, 4)
\numberwithin{figure}{section} % Number figures within sections (i.e. 1.1, 1.2, 2.1, 2.2 instead of 1, 2, 3, 4)
\numberwithin{table}{section} % Number tables within sections (i.e. 1.1, 1.2, 2.1, 2.2 instead of 1, 2, 3, 4)

\setlength\parindent{0pt} % Removes all indentation from paragraphs - comment this line for an assignment with lots of text

%----------------------------------------------------------------------------------------
%	TITLE SECTION
%----------------------------------------------------------------------------------------

\newcommand{\horrule}[1]{\rule{\linewidth}{#1}} % Create horizontal rule command with 1 argument of height

\title{	
\normalfont \normalsize 
%\textsc{università della vita \& youtube} \\ [25pt] % Your university, school and/or department name(s)
\vspace{5cm}
\horrule{0.5pt} \\[0.4cm] % Thin top horizontal rule
\huge \textsc{coin flip} \\ % The assignment title
\horrule{0.5pt} \\[0.4cm] % Thick bottom horizontal rule
}


\author{\small\textsc{fr}} % Your name
\date{\small\today} % Today's date or a custom date



\begin{document}
	

\maketitle

Diciamo che ho $N$ lanci di moneta: $p=1/2 = 1-p$. Mi chiedo quale sia il valore di aspettazione per $k$ teste o croci consecutive. Diciamo che $N=100$ e $k = 5$. Quante volte mi aspetto di trovare cinque teste o cinque croci in cento lanci? La soluzione analitica è semplice poiché la consecutività è in realtà un \emph{non-dato} del problema, in quanto l'ordine delle monetine è completamente irrilevante, infatti date quattro teste consecutive, qual è la probabilità che la prossima sia testa? Sempre $1/2$. La consecutività però non è un concetto insensato, e la risposta al quesito è data da: \emph{se pesco k monetine a caso nella stringa lunga N, qual è la probabilità che siano tutte uguali?} Poiché è indifferente che siano tutte testa o tutte croce il risultato va diviso per due:\[
\frac{1}{2} P(\text{k tutte uguali}) = \frac{1}{2} 2^{-k} = 2^{-k-1} = \frac{e^{-k\log(2)}}{2} \approx 0.5 e^{-0.69314718056\text{ } k}.
\]

Una simulazione numerica con $N=10^8$ (tabella \ref{tab:template}) mi tranquillizza sulla correttezza del risultato.

Se questa statistica rappresentasse un processo di decadimento, la vita media dell'isotopo in questione sarebbe l'unità di tempo.

\begin{table} 
	\centering 
	\begin{tabular}{l l l l } 
		\toprule
k & simulazione& besfit espo & formula \\
\midrule
0 & 0.49993551 & 0.49981028 & 0.50000000 \\ 
1 & 0.24998395 & 0.25012995 & 0.25000000 \\ 
2 & 0.12508653 & 0.12517748 & 0.12500000 \\ 
3 & 0.06247552 & 0.06264504 & 0.06250000 \\ 
4 & 0.03127232 & 0.03135070 & 0.03125000 \\ 
5 & 0.01560425 & 0.01568945 & 0.01562500 \\ 
6 & 0.00783274 & 0.00785178 & 0.00781250 \\ 
7 & 0.00390803 & 0.00392942 & 0.00390625 \\ 
8 & 0.00195316 & 0.00196648 & 0.00195313 \\ 
9 & 0.00097463 & 0.00098412 & 0.00097656 \\ 
10 & 0.00048932 & 0.00049250 & 0.00048828 \\ 
11 & 0.00024170 & 0.00024647 & 0.00024414 \\ 
12 & 0.00012059 & 0.00012335 & 0.00012207 \\ 
13 & 0.00006114 & 0.00006173 & 0.00006104 \\ 
14 & 0.00003036 & 0.00003089 & 0.00003052 \\ 
15 & 0.00001468 & 0.00001546 & 0.00001526 \\ 
16 & 0.00000758 & 0.00000774 & 0.00000763 \\ 
17 & 0.00000406 & 0.00000387 & 0.00000381 \\ 
18 & 0.00000212 & 0.00000194 & 0.00000191 \\ 
19 & 0.00000084 & 0.00000097 & 0.00000095 \\ 
20 & 0.00000040 & 0.00000049 & 0.00000048 \\ 
21 & 0.00000034 & 0.00000024 & 0.00000024 \\ 
22 & 0.00000018 & 0.00000012 & 0.00000012 \\ 
24 & 0.00000004 & 0.00000003 & 0.00000003 \\ 

		\bottomrule % Bottom horizontal line
	\end{tabular}
	\caption{il bestfit è del tipo $A e^{-B k}$, con $A=0.49981028$ e $B=0.69224800$} % Table caption, can be commented out if no caption is required
	\label{tab:template} % A label for referencing this table elsewhere, references are used in text as \ref{label}
\end{table}

%\horrule{0.1pt}
%\tableofcontents
%\horrule{0.1pt}
%
%%%%%%%%%%%%%%%%%%%%%%%%%%%%%%%%%%%%%%%%%%%%%%%%%%%%%
%\clearpage


\end{document}